%
%                       This is a LaTeX 2e version of the
%                       laboratory project template file.
\documentclass[a4paper,12pt]{article}
\usepackage{fullpage,epsf}


%
%                       This section generates a title page
%                       Edit only the sections indicated to put
%                       in the project title, your name, supervisor,
%                       project length in weeks and submission date
%
\begin{document}
\pagestyle{empty}                       % No numbers of title page                      
\epsfxsize=40mm                         % Size of crest
\begin{minipage}[b]{110mm}
        {\Huge\bf School of Physics\\ and Astronomy
        \vspace*{17mm}}
\end{minipage}
\hfill
\begin{minipage}[t]{40mm}               
        \makebox[40mm]{
        \epsffile{./images/crest.eps}}
\end{minipage}
\par\noindent                                           % Centre Title, and name
\vspace*{2cm}
\begin{center}
        \Large\bf \Large\bf Senior Honours Project\\
        \Large\bf Computational Physics\\[10pt]                     % Change to MP/CP/Astro
        \LARGE\bf Analyzing a ranking system         % Change to suit
\end{center}
\vspace*{0.5cm}
\begin{center}
        \bf Justs Zarins\\                           % Repace with your name
        March 2013                                    % Submission Date
\end{center}
\vspace*{5mm}
                   
\begin{abstract}
       %The abstract is a short, concise explanation of the project
%covering the aims, outlines of techniques used and a short
%summary of the results. It should contain enough information to
%make the aims and success of the project clear, but contain no details.
%A typical abstract should be between 50 and 100 words.
Simulations and data analysis were used to examine and evaluate the ranking system of British Orienteering. Generated and real data produced plausible rankings giving confidence to believe that the system is generally accurate and independent of initial setup. An issue of the rank distribution shrinking during repeated run-throughs of race data remains unresolved and impedes understanding of the system's convergence.
\end{abstract}

\vspace*{1cm}

\subsubsection*{Declaration}

\begin{quotation}
        I declare that this project and report is my own work.
\end{quotation}

\vspace*{2cm}
Signature:\hspace*{8cm}Date:

\vfill
{\bf Supervisor:} Prof. Graeme Ackland                % Change to suit
\hfill
10 Weeks                                         % Change to suit
\newpage
%
%                       End of Title Page
\pagestyle{plain}                               % Page numbers at bottom
\setcounter{page}{1}                            % Set page number to 1
\tableofcontents                                % Makes Table of Contents
\newpage

\section{Introduction}

The introduction section of the report should introduce the project in
more detail than in the abstract. In particular it should present the
motivation, the aims, outline of techniques used, and the scope of the project. 
It should also contain references to similar work in the
same field to put your work in the correct context.

As a general rule, people reading the abstract and introduction alone
should have a good idea of the material in the project, the techniques
employed and the results obtained. A typical introduction should be
about 1 page, (300-450 words). 
%Theory and background
% short overview of theoretical background to project
% principles of experiment, no derivations

%Literature survery
% a review of relevant topics and work within the field (not just a list)
% some depth
% must provide added value beyond results of report
% annotated well
% avoid webpages if possible

%\section{Background or Theory}


%Description of
% apparatus (code) and how it works
% experimental method and procedures
% calibration(?)

%Scope
% enough to allow th reader to udnerstand how the experiment was carried out
% very important that for people attempting to reproduce your results

%Useful tips
% diagrams!
% reference borrowed figures

%This section should contain the details of the method employed. 
%As in the previous sections standard techniques should not be written
%out in detail. For example if you use an oscilloscope to take a
%measurement, the theory of the CRO tube\footnote{Don't laugh, I have actually
%seen this.} is {\bf not relevant}. In computational projects this
%section should be used to explain the algorithms used and the layout of
%the computational code. A copy of the acutal code must be
%given in the appendices. Long detailed sections of theory, data tables
%and details of computational code used in data analysis only should not
%appear in this section, but should/may be included in the appendices.
%
%This section should emphasise the philosophy of the approach used
%and detail novel techniques. However
%please note: this section in {\bf not} a blow-by-blow account of what
%you did throughout the project, and in particular it should {\bf not} 
%contain large detailed sections about things you tried and found to be
%completely wrong. Remember you are writing a technical report, and
%not a diary. If however you find that a technique that was expected to
%work failed, that is a valid result and should be included.
%
%Here logical structure is particularly important, and you may find that
%to maintain good structure you may have to present the experiments
%in a different order from the one in which you carried them out.

\section{Method or Strategy}
All simulations and analysis scripts are implemented in Python 2.7. The Numpy library is used for certain numerical tasks and Pyplot is used for plots and histograms. 

\subsection{Simulation}
All simulation code follows the following pattern.
\begin{itemize}
\item Create a collection of runners and assign initial scores.
\item (optional) Assign each runner ability.
\item Execute the update loop m times.
	\begin{enumerate}
	\item Select a group of runners from the whole list.
	\item Generate run times.
	\item (optional) Mark slowest 10\% to be excluded from calculations.
	\item Calculate and apply score changes as per formula [???].
	\end{enumerate}
\end{itemize}

Two types of score initialization were considered: all starting from the same score of a 1000 or assign scores based on a a Gaussian peaking at 1000 and with standard deviation of 200. In the case of uniform score assignment care needs to be taken in the code to handle the case when SP is zero (all runners have the same score) leading to division by zero.
The code was progressively expanded to add more details about the runners. Initially there was no distinction between the competitors, later each runner was assigned an intrinsic ability and variability of that ability. The underlying ability distribution is modeled as each runner having a mean time. In reality, courses have different lengths, hence they take a different time to complete. The model used here corresponds to different subsets of all competitors running the same track in their own mean time. This should not be an issue because run times are essentially normalized when scores are calculated because only deviations from the mean time are relevant.
The variability of a runner's performance is represented by a mean number of mistakes they make in a race. Every time they compete, a Poisson random variable with a mean of that runner's specific number of mistakes is drawn. This is multiplied by a mistake weight factor and added to their run time. [+- mistakes as well...]

\subsection{Real Data Analysis}

\section{Results \& Discussion}

This section should detail the obtained results in a clear,
easy-to-follow manner. Remember long tables of numbers are just as boring to
read as they are to type-in. Use graphs to present your results where
-ever practicable. When quoting results or measurements
{\bf DO NOT FORGET ABOUT ERRORS}. Remember there are two basic types
of errors, these being random and systematic, which you must consider.
Remember also the difference between an error and a mistake, computer
program bugs are mistakes.

 
Again be selective in what you include. Half a dozen
tables that contain totally wrong data you collected while you forgot
to switch on the power supply are {\bf not relevant} and will frequently
mask the correct results. 
%
%                       Here is how to inserted a centered
%                       postscript file, this one is actually
%                       out of Maple, but it will work for other
%                       figures out of Xfig, Idraw and Xgraph
%
\begin{figure}[htb]     %Insert a figure as soon as possible
        \begin{center}
                \leavevmode             % Warn Latex a figure is comming
                \epsfxsize=90mm         % Horizontal size YOU want
                                        % figure to be
                \epsffile{./images/otf.eps}
\end{center}
\caption{This is an inserted Postscript file}
\end{figure}

This section must contain a discussion of the results. This should
include a discussion of the experimental and/or numerical errors, and a
comparison with the predictions of the background and theory underlying
the techniques used. This section should highlight particular strengths
and/or weaknesses of the methods used.
%objective achieved?
%summary of main results
%significance, relevance, confidence (of/in results)
%future work

\section{Conclusion}
This section should summarise the results obtained, detail
conclusions reached, suggest future work, and changes that you would make if you repeated the
experiment. This section should in general be short, 100 to 150 words
being typical for most projects.
\par\noindent
If you have opted to have multiple {\bf Theory, Method, Results}
sections, draw all the results together in a {\bf single} conclusion.

 

\newpage
\addcontentsline{toc}{section}{References}
\bibliographystyle{unsrt}
\bibliography{bibliography}
      
\appendix
\section{Appendices}

Material that is useful background to the report, but is not essential,
or whose inclusion within the report  would detract from its
structure and readablity, should be included in appendices. Typical
material could be diagrams of electronic circuits built, specialist
data tables used to analyse results, details of computer programs
written for analysis and display of results, photographic plates,
and, for computational projects, a copy of all written code.

Again be selective. The appendix is {\bf not} an excuse for you to add every
last detail and piece of data, but should be used to assist the reader
of the report by supplying additional material. Not all reports require
appendices and if the report is complete without this additional
material leave it out.

\end{document}
