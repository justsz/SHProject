\section{Method or Strategy}

This section should contain the details of the method employed. 
As in the previous sections standard techniques should not be written
out in detail. For example if you use an oscilloscope to take a
measurement, the theory of the CRO tube\footnote{Don't laugh, I have actually
seen this.} is {\bf not relevant}. In computational projects this
section should be used to explain the algorithms used and the layout of
the computational code. A copy of the acutal code must be
given in the appendices. Long detailed sections of theory, data tables
and details of computational code used in data analysis only should not
appear in this section, but should/may be included in the appendices.

This section should emphasise the philosophy of the approach used
and detail novel techniques. However
please note: this section in {\bf not} a blow-by-blow account of what
you did throughout the project, and in particular it should {\bf not} 
contain large detailed sections about things you tried and found to be
completely wrong. Remember you are writing a technical report, and
not a diary. If however you find that a technique that was expected to
work failed, that is a valid result and should be included.

Here logical structure is particularly important, and you may find that
to maintain good structure you may have to present the experiments
in a different order from the one in which you carried them out.