\section{Results \& Discussion}

This section should detail the obtained results in a clear,
easy-to-follow manner. Remember long tables of numbers are just as boring to
read as they are to type-in. Use graphs to present your results where
-ever practicable. When quoting results or measurements
{\bf DO NOT FORGET ABOUT ERRORS}. Remember there are two basic types
of errors, these being random and systematic, which you must consider.
Remember also the difference between an error and a mistake, computer
program bugs are mistakes.

 
Again be selective in what you include. Half a dozen
tables that contain totally wrong data you collected while you forgot
to switch on the power supply are {\bf not relevant} and will frequently
mask the correct results. 
%
%                       Here is how to inserted a centered
%                       postscript file, this one is actually
%                       out of Maple, but it will work for other
%                       figures out of Xfig, Idraw and Xgraph
%
\begin{figure}[htb]     %Insert a figure as soon as possible
        \begin{center}
                \leavevmode             % Warn Latex a figure is comming
                \epsfxsize=90mm         % Horizontal size YOU want
                                        % figure to be
                \epsffile{./images/otf.eps}
\end{center}
\caption{This is an inserted Postscript file}
\end{figure}

This section must contain a discussion of the results. This should
include a discussion of the experimental and/or numerical errors, and a
comparison with the predictions of the background and theory underlying
the techniques used. This section should highlight particular strengths
and/or weaknesses of the methods used.