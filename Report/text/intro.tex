%The introduction section of the report should introduce the project in
%more detail than in the abstract. In particular it should present the
%motivation, the aims, outline of techniques used, and the scope of the project. 
%It should also contain references to similar work in the
%same field to put your work in the correct context.
%
%As a general rule, people reading the abstract and introduction alone
%should have a good idea of the material in the project, the techniques
%employed and the results obtained. A typical introduction should be
%about 1 page, (300-450 words).
\section{Introduction}
%
Objectively comparing and ranking a set of entities is an important challenge. Consumers are daily tested to choose the right products, companies look for the most fruitful strategies to pursue, online search engines order query results by relevance and competitors seek to find who is the best. If there are clear criteria for comparison, ranking is straightforward. For example, marathons are always the same distance and with similar surface properties and topology. Some additional details may need to be taken into account like different age groups of the runners, but in general it is sufficient to compare the competitors' run rimes (historic or recent) in order to establish their ability.\\ 
However, not all ranking problems present clear criteria. In team sports and multi-player video games the landscape is more complicated. How do individual player contributions accumulate to reach the eventual victory or defeat? Is a team that barely wins or one that wins by a large margin better in the long run? These questions hardly have a conclusive answer. There are also added complications because people act differently depending on the nature of the competition. For instance, students in exams generally attempt to do their best without directly knowing how well their competition is doing, yet in the end it is their relative rather than absolute performance that matters most. In contrast, a marathon runner might go for their personal best, try to set a world record or just stay a few seconds ahead of their pursuer. These and other approaches reflect a person's underlying ability differently.\\
One way to approach complicated systems is to abstract away from specific details and employ statistical methods; increasing sophistication does not always produce increasingly accurate results. A ranking scheme of this type is employed by the British Orienteering Federation (BOF) to assign points to orienteering event participants. The aims of this project is to evaluate the BOF ranking scheme for accuracy and to uncover the cause of some of its quirks like the drifting of the mean of ranks over time. The problem will first be approached using computer simulated races and later the scheme will be applied to real race data.\\