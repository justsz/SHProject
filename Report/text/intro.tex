%The introduction section of the report should introduce the project in
%more detail than in the abstract. In particular it should present the
%motivation, the aims, outline of techniques used, and the scope of the project. 
%It should also contain references to similar work in the
%same field to put your work in the correct context.
%
%As a general rule, people reading the abstract and introduction alone
%should have a good idea of the material in the project, the techniques
%employed and the results obtained. A typical introduction should be
%about 1 page, (300-450 words).

\section{Introduction}

Objectively comparing and ranking some set of entities is an important challenge. Consumers are daily tested to choose the right product, companies look for the most fruitful strategies to pursue, online search engines order query results by relevance and competitors seek to find who is the best. If there are clear criteria for comparison, ranking is straightforward. For example, marathons are always the same distance and with similar surface properties and topology. Some additional details may need to be taken into account like different age groups of the runners, but in general it is sufficient to compare the competitors' run rimes (historic or recent) in order to establish their ability. 
However, not all ranking problems present clear criteria. In team sports and multi-player video games the landscape is more complicated. How do individual player contributions accumulate to reach the eventual victory or defeat? Is a team that barely wins or one that wins by a large margin better in the long run? [......]
One way to approach such systems is to abstract away from specific details and employ statistical methods. This report will examine the ranking scheme used in British orienteering competitions through computer simulation of ideal data. The scheme will then be applied to real race data and improvements will be sought.




%Theory and background
% short overview of theoretical background to project
% principles of experiment, no derivations

%Literature survery
% a review of relevant topics and work within the field (not just a list)
% some depth
% must provide added value beyond results of report
% annotated well
% avoid webpages if possible

\section{Background or Theory}

